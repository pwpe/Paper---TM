\section{Introduction}

In response to ageing societies, there is an increasing need for people to take an active role in their own health and well-being. Designing technologies that support self-reflection, awareness and self-management of chronic conditions have particularly been of interest to HCI researchers to support quality of life issues. Chronic Obstructive Pulmonary Disease (COPD) is a progressive lung disease in which the airways are damaged. People with chronic health conditions such as COPD occasionally experience exacerbations that, if not detected and treated early, result in an increased use of healthcare services and a decline in health-related quality of life. Previous studies have shown poor self-management among COPD patients, who do not respond to the early warning signs, because they have difficulties recognizing the onset of an exacerbation or its importance, resulting in delayed recognition and treatment of exacerbations. 

Telehealth technologies where patients log and keep track of important variables and symptoms (e.g. self-reported shortness of breath or fluctuations in oxygen saturation measures) are widely used today by healthcare providers to remotely monitor patients and support early detection and initiation of treatment. We refer to the ongoing activity of logging data (objective or subjective) during concrete \textit{episodes} over time as \textit{self-tracking}. Self-tracking is well-understood from literature on the Personal Informatics movement, where people voluntarily, self-initiate and successfully use technology to enhance self-awareness, self-reflection and thereby change behaviours, improve health or other aspects of life. While the health literature provides little information on, how patients are supported in their self-tracking efforts in the telehealth context, Personal Informatics literature investigates interaction design aspects of self-tracking important for the successful integration of technology in self-tracking interventions. 

We synthesized findings from the disparate fields of telehealth and Personal Informatics and used it as an analytical lens for material from semi-structured interviews with COPD patients using a telehealth solution to understand, how we can support patients in their self-tracking efforts (Study 1). While patients generally felt safe, many struggled with subjective data entries that employed references to perceived baselines. The state of the art system they used provided little support on follow-up actions and reflection by excluding access to historical data. 

We used findings from the literature and the study to propose design solutions through mockups, focusing on supporting self-reflection that could potentially support patients in the early detection and treatment of exacerbations. We conducted individual feedback sessions on design mockups with five COPD patients (Study 2) and found that not all patients were willing to engage in self-reflection and identified some major concerns related to the design proposals. 

We redesigned and implemented a prototype to evaluate it in a real context during a two-weeks trial by six new COPD patients (Study 3). The trial informed, how four concrete design decisions affects reflection among COPD patients. This paper presents key findings on how telehealth systems in the future may be designed to support reflection, finish by drawing some key concerns for those intending to design and evaluate reflection. 