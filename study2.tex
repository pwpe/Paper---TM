In the following we describe how we redesigned the system to support patients in collecting and reflecting on self-tracked data and evaluated the re-design. Researchers have however found that co-designing with COPD patients using generative tools and techniques (e.g. post-it notes and sketching activities) are resource demanding for the patients, considering their health condition. Some patients are not even able to participate in other activities than keeping a conversation. In Study 1, we similarly experienced that COPD patients are physically limited (e.g. in terms of moving from one place to another) and within an hour of interview experienced breathing difficulties several times, demanding a slow pace and long pauses. 

Drawing from Das et al.’s experiences on co-designing with COPD patients \cite{Das} and our own experiences from Study 1, we decided not to conduct co-designing activities as initially planned. We made a conscious plan to conduct individual sessions where patients were only expected to talk or interact with a prototype. To further reduce time and effort required by them in the sessions, we asked patients to complete workbook assignments beforehand, allowing them to prepare for the discussion on, how our proposals foster reflection during collection and review of data. In the following, we describe the workbook and prototype design. 

\subsection{Workbook Design}
We found that patients reflect during multiple stages of a collection episode and have user needs and concerns, currently not addressed in the system. Workbooks assignments were designed to get insight into, how the system can support them in collecting and reflecting on data. We further asked patients to collect data to be used in the prototype. 

We found that patients were unsure, whether they were collecting data under the right conditions in Study 1. To get further insight into, how patients reflect on conditions relevant for the validity and reliability of their measures, we asked them to annotate if they had reflected on context-relevant variables that we provided in the workbook. For dyspnea measures, context-relevant variables could be e.g. weather, mood, smoking or physical activity. Patients were asked to comment on it as free text annotation one day and mark it in checkboxes another day (e.g. mark with X if medicine has affected saturation measure). 

Patients expressed difficulties rating symptoms subjectively, due to lack of baseline understanding and low granularity of answer options in Study 1. Patients had to reflect on a time series graph of previous measures (showing baseline) while entering current measure in one assignments. In another assignments, patients had to assess dyspnea on different scales: (1) Relative to usual baseline (binary), (2) absolute (five-point scale) and (3) using visuals (Dalhousie Pictorial Scale \cite{dalhousie}).  

Finally, patient had to reflect on a time series graph of previous oxygen saturation measures (long-term reflection). A recommended level was marked on the graph to trigger reflection on a potential mismatch in accordance with Festinger’s cognitive dissonance theory \cite{Rivera}. 

\subsection{Prototype Design}
\subsubsection{Improving Reliability and Validity of Measures}
Identical to the reference graph in the workbook, we also incorporated a time series graph visualizing patients’ previous measures in the prototype to support patients in remembering previous events while entering data. We changed symptom rating options from binary to scale-based, inspired by the questionnaire-items in EXACT PRO, a validated tool to measure COPD exacerbations through subjective assessments of symptoms \cite{exact}. 

Similar to the assignment in the workbook, we provided the option to toggle context-relevant variables for each measure. These were also introduced in the prototype to support reflection described in the next section. 

\subsubsection{Supporting Short-term and Long-term Reflection}
From literature, we found that supporting both short-term and long-term reflection can be important, depending on the conditions in which the user might self-track \cite{Li2010, Muller}. 

To support short-term reflection, we provided patients with a dashboard view after collection (See Figure ). The dashboard view allows patients to reflect on current status immediately after collection and increase awareness on their current status \cite{Cutone, Muller}. We included reflective questions to trigger reflective description (R1) or higher levels of reflection on this view and to encourage users to further explore their data \cite{Fleck, Muller}. E.g. \textit{“Why are you coughing more than last time you measured?”}. Gauges for each measure show current measure in relation to the recommended level (goal) and arrows indicate change from previous measure (day-to-day variations). The system uses color indications (red, yellow and green), where red or yellow highlight a potential mismatch and thereby actively trigger reflection in accordance with Festinger’s cognitive dissonance theory \cite{Rivera}.  

To support long-term reflection, we designed four different visualizations (See Figure ). (V1) Time series graphs with option to compare two measures, (V2) Time series graphs stacked vertically, (V3) Calendar heatmap and (V4) Area graphs stacked vertically. V1, V2 and V3 were proposed to foster reflection on upwards and downwards trends or symptoms deviations to increase awareness on a worsening in condition \cite{Rivera}. V1 differed in that it allowed for comparison of multiple measures that could trigger reflection on how measures are related or change over time \cite{Cuttone}, affording dialogic reflection (R2) \cite{Fleck}. Recommended levels were marked on all three visualizations to increase awareness on discrepancies \cite{Li2010} and trigger reflection. V3 was proposed as an alternative to time series graphs as mentioned by Cuttone et al. to support reflection on periodic patterns using color shades to indicate deviations from recommended level \cite{Cuttone, Li2010}.   

To further support long-term reflection, we provided the option to toggle on discrete events on the visualizations (context-relevant variables) \cite{Sorensen}. This feature was included to support reflective description (R1) and dialogic reflection (R2) by allowing for exploration of relationships between context-relevant variables and measures otherwise invisible \cite{Fleck}.   

\subsection{Participants and Method}
We asked the same patients as in the previous study to participate in feedback sessions on the redesigned system. All patients except P4 who was hospitalized participated. Three spouses (P2S, P5S and P6S) also participated. Feedback sessions were held in the patient’s home and lasted for approximately one hour. 

Patients completed workbook assignments three days a week (Monday, Wednesday and Friday) delivered one week prior to the feedback sessions. The feedback session was divided into a workbook session and a prototype session, where one facilitator and an observer were present. The observer initially took the facilitator role and asked patients follow-up questions emerged from Study 1, while the facilitator went through the workbooks and prepared for the feedback session. In the workbook session, the facilitator interviewed and discussed the assignments patients had completed in the workbook. In the prototype session, the facilitator guided the patients through each screen in the prototype and asked them to think-aloud and complete tasks. The facilitator encouraged patients’ spouses to participate in in the discussion and provide their points of view. The prototype was adjusted to use patient’s own values for visualizations (collected in workbooks) to encourage reflection on patients’ own experiences and not fictive values. The observer prepared the prototype during the workbook session. The sessions were audio-recorded. 

\subsection{Results}     

\subsubsection{Reliability and Validity of Measures}
Not all patients were consistent in terms of \textit{when} and \textit{how} they took measures. While P2 and P5 ensured taking measures under comparable conditions (always in the morning), P1, P3 and P6S lacked knowledge on relevance of context and how some of the provided context variables influence their measures. E.g. P1 and P6S questioned whether and why a cold finger when taking oxygen saturation measures has an influence. 

Some context variables are more likely to affect submitted measure than others. E.g. in case of shaking hand, P6 and P6S retake oxygen saturation measure until it is stable. Patients mentioned additional context variables relevant for their measures (e.g. mood, talk, supplemental oxygen). P1 only weighed himself every three weeks, despite the system prompts him to enter current reading. P2 and P6S specifically asked for guidelines on taking measures. \textit{“For some measures, an explanation would be good. For example do not take measure if this and that“} (P6S). 

All patients preferred higher granularity options when rating symptoms more than binary. P5 and P6 mentioned that they preferred higher granularity, because it shows more and makes it possible to show degrees. “How much is a no? If we say yes or no to the hospital, they still do not know what we are thinking .. then they’ll call us and we’ll still have to say it is severe” (P6S). 

Asking to rate symptoms without a baseline comparison (“Did you feel breathless today?”) and having five answering options (not at all, slightly, moderately, severely, extremely) caused difficulties for some patients, because they experienced several of the options during the day. \textit{“I have been through all of the provided options that day. How do you want me to answer that?“} (P5). Patients further had different perceptions of what usual is, when asking to rate relative to usual baseline. \textit{“(..) I base that on when I’m at my best”} (P3), \textit{“usual is when it is an ordinary day”} (P5) and \textit{“If she is not more breathless than yesterday, then we’ll just submit a no”} (P6S). 

Patients presumed that it would not be of benefit to them to have a time series graph showing their baseline. “I do not think it has any effect to see a graph” (P3). P2 mentioned, “I shouldn’t answer based on what I answered last time. I should answer what it is now and here”, but after discussing its purpose agreed that it would help him remember previous measures. 

\subsubsection{Short-term and Long-term Reflection}
In the feedback session, some patients were not willing to reflect on visualizations of current status, reflective questions or history data. E.g. “I do not care what my status is.. I just submit data.. do not walk around and think everyday .. I know how I am feeling” (P3). P5 expressed interest in current status and what action should be taken, “I’m more concrete. Where am I right now and what can I do about it?” (P5). The dashboard view helped some patients in getting an overview of their health status. P5 thought that the gauges on the dashboard illustrated her status quickly. P2 mentioned that the arrows helped him \textit{“quickly see if it (a measure) is going up or down”}. Reflective questions were not noticed by the patients. When asked explicitly to comment on them, some patients seemed aware of the answers \textit{“right now it is likely because I talk too much”} (P5). 

None of the patients expressed interest or benefits in having access to previous measures prior to the feedback session, e.g. “I do not need it (access to history data)” (P3). Several patients relied on the healthcare monitoring “I have a nurse who is good at keeping an eye on me” (P5), “I do not need it (access to history data). If it (measure) is too low, they call and ask me why” (P1). P2 similarly expressed that he did not know have the necessary knowledge to find it useful “I do now know what they use it for, the scales they use and the language.. I do not understand it. I count on they react if there is anything” (P2). 

Providing patients with a recommended level that they could compare their measures against, provoked negative feelings among some: “I prefer not to be told in the morning that I’m gonna get an awful day” (P5), “It’s ok if it’s just a single measure, but if it is constant, I would start thinking.. It’s going fast now” (P2). While P5 mentioned that she would ignore the recommended level, because it did not match with her own goal, \textit{“I thought, you can forget it (about provider-recommended level). I’ll just do what I usually do”}, P2 suggested that he would strive to keep his measures within the recommended level, “then it’s not that bad if I keep it above that (lower threshold)”. P6S indicated that if his wife’s oxygen saturation was above normal area, he would start wonder whether the oxygen supply was set too high and initiate action. \textit{“If it starts to go over here (below normal area), we have to do something”} (P6S). 

Despite none of the patients initially expressed benefits in having access to history data, two patients (P2 and P5) changed their attitude after the prototype session. “This gives more information about me (...) it’s nice to be able to go back.. Is it better than 14 days ago?” (P2) and “There might be days where I sit with it and have an idea about what I’m looking for, which might trigger some thoughts” (P5). Others were more reluctant on reflecting. P3 did not see any benefits and found it troublesome, while P1 did not think it was his job to look at history data, \textit{“this is only for people who has to sit and analyse the numbers”} (P1). V1 afforded finding relations between measures e.g. by comparing, “you can have them (measures) together and see how they affect one another” (P2). V3 was more attractive to others, who found it more concrete and provided a quick overview, \textit{“it (V3) is the one I get the quickest.. If you are in doubt what the colors mean, you can see them down there”} (P5). PS6 thought that he would start with V2 and then use V1 when he had become more advanced. 

\subsection{Discussion}
We identified two types of patients in this study: Active and Passive. Passive patients (P1 and P3) had met one or several barriers in terms of reflection. They were not willing to reflect on their data, because they did not see any purpose for reflecting. While we know from Study 1 that all patients were motivated to collect data, because having a healthcare professional monitoring provided them with a sense of security, it became clear in this study that passive patients were not equally motivated to reflect on the data. These patients took the role as data providers and did not see any benefits in long-term reflection. Active patients (P2 and P5) were more open-minded towards engaging in the discussion of supporting reflection in the system. These patients could potentially benefit short-term and long-term reflection support in the system.

Active patients explicitly mentioned being aware of taking measures under comparable conditions, while passive patients were not. Patients were aware that the self-tracked measures they submit are affected by context data. Providing an option in the system to annotate context-related variables could potentially support reflection on measures in later review. Some patients lacked knowledge about specific context data and requested guidelines in the system. 

The dashboard view provided information about patients’ current status. Reflective questions were not noticed by patients initially. In passive patients reflective questions did not trigger any reflection (R0), while active patients proposed lower level explanations as answers (R1). 

Patients had different preferences in terms of visualization for long-term reflection. One patient mentioned needing a purpose and time to reflect for gaining any benefits from history data, in accordance with Fleck \& Fitzpatrick \cite{Fleck}. None of the patients were able to express, how the visualizations for long-term reflection supported reflection, except V1, which one patient mentioned could support identifying relationships between measures. This visualization, however, was difficult for several patients who did not see the purpose of it or had usability problems. Having a recommended level to compare measures with, provoked negative feelings among some patients, e.g. becoming worried. In some active patients it supported reflecting on action, if measures were not hypothetically within the recommended area. 

Assignments in the workbook and prototype did not elicit many findings on patients’ reflective thoughts on the design proposals. While we did use patients’ own values rather than fictive values to discuss the workbook and prototype assignments in terms of long-term reflection, we used fictive values on the dashboard, which could have been a barrier for reflection in some patients. Additionally, patients might have found it difficult to reflect, because (1) the interview period was time-limited and (2) a reason (e.g. a worsening in health status) was needed to trigger reflection. 

\subsection{Conclusion}
We evaluated a re-designed telehealth system that aimed at improving collection of reliable and valid data and supporting short-term and long-term reflection. We found that annotating context-relevant variables to measures in the collection stage is relevant and could potentially support reflection in later review. Passive patients were not willing to reflect, but active patients could benefit from support in short-term and long-term reflection. 