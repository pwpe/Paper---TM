\section{Study 1 - Exploring User Needs in Telehealth}
In the following study we explore self-tracking needs and concerns in telehealth context using COPD patients as our case. 

\subsection{Participants and Method} 
Six COPD patients (P1-P6), two male patients (P1, P2) between 64 and 65 years (M: 64.5) and four female patients between 54 and 74 years (M: 66.8), participated in the study. All patients had either severe or very severe COPD and multiple other health-related conditions (diabetes, heart disease, pulmonary oedema, asthma, bronchitis, osteoporosis and sleep apnea). Three of the patients used supplemental oxygen (P3, P5, P6). All of them lived in their own homes with a spouse, except P3 who lived alone. P6’s spouse (P6S) was her spokesman, as she had a speech disorder. 

We conducted audio-recorded semi-structured interviews in patients’ own homes using an interview guide as the framework for discussion (See Worksheets). The participants demonstrated to an interviewer and a researcher who took field notes, how they used a telehealth system AmbuFlex (AF). Participants had used AF in between three months and two years. Three of the participants had previously used Tunstall HealthCare (THC) solution (for six months). 

AF is web-based and can be accessed by mobile or desktop. THC consisted of a monitoring box installed in the patient’s home. In both systems, patients submitted objective data (oxygen saturation, lung function (only in THC), pulse and weight) and subjective data (binary answers to whether dyspnea, cough and sputum color had been higher than usual) three times a week (Monday, Wednesday and Friday). AF provided confirmation of submission but no option for reviewing previous data. Patients received follow-up calls when healthcare providers monitoring the data needed additional information (e.g. in order to discuss deviations and/or advise patients’ to initiate medication) or validation of measures. 

We went through the field notes and audio-recordings after the interviews. In the following we present, how patients engaged in the different stages of self-tracking using AF. 

\subsection{Results}
All patients remembered to take their measurements consistently and routinely in the morning themselves (P6S took responsibility for P6). 

\paragraph{Preparation:} 
The majority of participants (P1-P5) found the sense of security from healthcare providers monitoring their data motivating. “\textit{(...) it gives you a huge sense of security that you are not gonna lay at home ill}” (P2). P4 felt obligated to take measurements due to the presence of a healthcare provider. 

Two patients tracked additional data on paper (P5, P6S). P5 used the data as documentation, e.g. when being admitted to the hospital to discuss it with healthcare providers. P6s spouse mentioned curiosity, self-satisfaction and sense of agency as motivations for tracking on paper. 

\paragraph{Collection:} 
Patients found data collection easy, not requiring expert computer skills, and not taking too much effort or time. P5 stressed the importance of fast collection, “\textit{it must not take ten or fifteen minutes to do it everyday (...)}“. “\textit{This [AF] is really simple (...) it is so simple you can add some more to it}” (P6S). Patient mentioned currently spending between 1-2 minutes and 15-20 minutes on AF. Several patients found answering subjective questions difficult when it required comparisons with the ‘usual’ baseline, (“Are you coughing more than usual?”). “\textit{What is usual? Isn’t that also how I felt yesterday? Otherwise, I have misunderstood the question}” (P4). Patients needed higher than binary granularity to answer, “\textit{When they ask if you have more dyspnea than usual, then we say yes .. but how much is it? They [healthcare providers] cannot see}” (P6S). Some patients underreported baseline deviations and only answered “yes” in large or extreme deviations, “\textit{if it’s just a little different, I do not mention it}” (P2),  “\textit {I would have to be coughing a lot and feel very ill, if I answer yes to that question (...)}” (P4). P2 asked for a scale instead “\textit{(..) why don’t they make a scale instead for example from 1 to 5 or 1 to 10? One day I could perhaps say it’s 5, the next day 6 and the day after I can go back to 5}”. P5 used the comment box to make small deviations go on record, “\textit{(...) to me it is important that we take every small nuance}”.  When to collect, was a concern for P4 whose oxygen saturation measure and pulse depended on her level of activity. She wondered why the system did not take into account external factors related to her condition, “\textit{Do you feel more breathless today? But it does not say anything about the fog outside}” (P4). 

\paragraph{Reflection:} 
Several patients mentioned that an exacerbation comes within a few hours or even minutes, and that they were not able to recognize an onset by using AF. P1, P2, P3 and P5 measured oxygen saturation several times a day to verify their subjective feeling of well-being (P2, P5) or lack thereof. None of the patients felt they learned anything about their disease using AF. \textit{I can feel it [an exacerbation], even if I did not have the monitoring device}” (P3). Patients did not express any concerns waiting for a (potential) call. Most of them had identified “the hours” of the reviewers at the hospital and several a mental model of when a call would ensue. “\textit{They usually do it [review the data] before noon}” (P3). “\textit{I already know when there is going to be a call (..) when the oxygen saturation is too low, the pulse is high and your measures fluctuate, they react}” (P2). P6S found benefits in tracking data on paper, allowing for understanding his wife’s baseline and whether she was deviating from it and getting worse. “\textit{You can see how stable it is .. (..) Let’s say she loses weight then I become alert that something is wrong}” (P6S). Patients had not been informed by their healthcare providers about their “normal range” (recommended level) and the AF interface did not communicate it either. Half the patients wanted to know these in numbers. Some of the patients had identified their own “normal range” of oxygen saturation that mapped to not feeling well (usually below 90). 

\paragraph{Action:} 
All patients had received education in self-management of their condition (e.g. breathing techniques), but not all patients gained the same benefits from it and needed actionable advice. To that end some patients (P2, P5) added questions to the comment box. P2 acted on the basis of his oxygen saturation measures, “\textit{when it [oxygen saturation measure] is lower than 93, you do not feel fine (...) then I walk a little slower and take it a bit more easy}”. P2 was interested in knowing additional methods to increase oxygen saturation. P6S wanted recommendations on duration for supplemental oxygen use based on her oxygen saturation measures and information about variables e.g. the weather, that could influence her symptoms. P3 and P4 used their oxygen saturation measures adjust their supplemental oxygen. However, P3 preferred not to initiate treatment including drugs before consulting a healthcare provider, unless in extreme cases of symptoms or unavailability of staff. “\textit{I might to do it [initiate medication treatment] if it [sputum color] was very green, if it was a Tuesday [a day not monitored by healthcare providers], otherwise I wouldn’t (...)}”.

\subsection{Discussion}
Our patients were highly motivated to track potentially due to the active role of the healthcare provider that provided them with a sense of security not present in previous studies \cite{Li2010, Ancker2015, Chung2015}. One patient relied on the monitoring to such an extent that she sometimes delayed treatment, waiting for confirmation from the healthcare provider.
 
None of the patients described the tracking activity requiring too much effort or time. However, effort seemed to be an aspect that should be considered when designing telehealth systems, as some patients both expressed willingness to spend more time than AF required (approximately two-three minutes), but not wanting to spend more than ten to fifteen minutes.

Based on our findings and literature review, we revised the \textit{Lived Informatics Model} \cite{Epstein2015}. We broke down a data collection episode into $pre-collection$ (deciding on whether to log or skip), $acquisition$ (ready required artifacts), \textit{calibration period} (satisfy guidelines for tracking), $measure$ (taking measure using artifact), $entry$ (entering read off measure from artifact or providing scale based ratings, absolute or relative to a baseline, or qualitative comments) and $submission$ (submitting data). The patients reflected during multiple stages of collection before entering both subjective and objective measures. 

AF did not meet the needs of users in terms of (1) scope, (2) reliability, (3) validity, (4) actionable advice. AF’s scope focused only on submitting variables directly related to the condition at time of entry, but two patients tracked additional and one logged data on paper (c.f. \cite{Patel2012, Chung2016}). Having access to their previous data made patients feel in control. 

In terms of reliability, some patients were unsure whether they were collecting data under the right conditions. Subjective questions with a baseline comparison proved difficult due to: No access to the baseline and the low granularity of the answer options. Patients had insufficient access to their usual subjective feelings and tried to remember previous events to establish their ‘usual’ baseline (c.f. \cite{piloting}) and AF provided no access to historical data. Even if AF provided access to previous data this might prove difficult due to the low granularity. The binary answer options resulted in reduced validity of data by underreporting significant increases from the baseline. One patient specifically asked for rating on a scale instead, which requires more cognitive effort and time \cite{Oh2015}. 
Due to the absence of data access, patients did not interact with the data they had collected and expressed not having learnt anything from telehealth. Several patients mentioned not being able to recognize onset of an exacerbation from AF use, suggesting that the system poorly supported reflection. 

Patients did not know their provider-recommended “normal range” and therefore used their own identified “normal range” in management of their condition using the pulse oximeter. Some patients wanted to know the provider-recommended “normal range” to become more empowered, while others were not interested. One reason for that could be that patients get reminded about the negative aspects of their health when reviewing data or that they rely more on their subjective feeling than on quantities, as in \cite{Ancker2015}. 

Several patients were interested in actionable advice from the system as in \cite{Chung2015, Li2010}. Apart from during subjective data entry, patients needed two types of support, (1) confirmation from healthcare providers to act (e.g. initiation of medication) and (2) actionable advice on self-management strategies (e.g. coping with breathlessness). We believe that one of the barriers to action was the lack of support for reflection during entry and review of data  - a prerequisite to action according to Li et al. \cite{Li2010}.

\subsection{Conclusion}
While having a healthcare provider monitoring data motivated sustained tracking, the telehealth system they currently used did not sufficiently meet user needs for tracking. Patients expressed difficulties rating their symptoms relative to their usual baseline and uncertainty in terms of which conditions to measure in, resulting in reduced reliability of the data. A lack of access to their historical data hindered patients in entering reliable data, reflecting and taking actions. 